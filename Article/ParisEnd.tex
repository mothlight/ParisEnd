%% 
%% Copyright 2007, 2008, 2009 Elsevier Ltd
%% 
%% This file is part of the 'Elsarticle Bundle'.
%% ---------------------------------------------
%% 
%% It may be distributed under the conditions of the LaTeX Project Public
%% License, either version 1.2 of this license or (at your option) any
%% later version.  The latest version of this license is in
%%    http://www.latex-project.org/lppl.txt
%% and version 1.2 or later is part of all distributions of LaTeX
%% version 1999/12/01 or later.
%% 
%% The list of all files belonging to the 'Elsarticle Bundle' is
%% given in the file `manifest.txt'.
%% 
%% Template article for Elsevier's document class `elsarticle'
%% with harvard style bibliographic references
%% SP 2008/03/01

%\documentclass[preprint,12pt,authoryear]{elsarticle}  %default in the template
%\documentclass[preprint,10pt,authoryear]{elsarticle}

%% Use the option review to obtain double line spacing
%% \documentclass[authoryear,preprint,review,12pt]{elsarticle}

%% Use the options 1p,twocolumn; 3p; 3p,twocolumn; 5p; or 5p,twocolumn
%% for a journal layout:
%% \documentclass[final,1p,times,authoryear]{elsarticle}
%% \documentclass[final,1p,times,twocolumn,authoryear]{elsarticle}
 \documentclass[final,3p,times,authoryear]{elsarticle}
%% \documentclass[final,3p,times,twocolumn,authoryear]{elsarticle}
%% \documentclass[final,5p,times,authoryear]{elsarticle}
%% \documentclass[final,5p,times,twocolumn,authoryear]{elsarticle}

%% For including figures, graphicx.sty has been loaded in
%% elsarticle.cls. If you prefer to use the old commands
%% please give \usepackage{epsfig}

%% The amssymb package provides various useful mathematical symbols
\usepackage{amssymb}
%% The amsthm package provides extended theorem environments
 \usepackage{amsthm}
 \usepackage{amsmath}
 \usepackage{color}
 \usepackage{amsmath}
\usepackage{siunitx}


\usepackage{framed} % Framing content
\usepackage{multicol} % Multiple columns environment
\usepackage{nomencl} % Nomenclature package
\makenomenclature
%\setlength{\nomitemsep}{-\parskip} % Baseline skip between items
\setlength{\nomitemsep}{0.01cm}
\renewcommand*\nompreamble{\begin{multicols}{2}}
\renewcommand*\nompostamble{\end{multicols}}
\newcommand{\degreeC}{\ensuremath{^\circ}C }

\usepackage[nonumberlist]{glossaries}
\makeglossaries 


%% The lineno packages adds line numbers. Start line numbering with
%% \begin{linenumbers}, end it with \end{linenumbers}. Or switch it on
%% for the whole article with \linenumbers.
%% \usepackage{lineno}

\journal{Urban Climate}

\newglossaryentry{aa}{name={$a$},symbol={\ensuremath{a}},description={a}}

\begin{document}

\begin{frontmatter}

%% Title, authors and addresses

%% use the tnoteref command within \title for footnotes;
%% use the tnotetext command for theassociated footnote;
%% use the fnref command within \author or \address for footnotes;
%% use the fntext command for theassociated footnote;
%% use the corref command within \author for corresponding author footnotes;
%% use the cortext command for theassociated footnote;
%% use the ead command for the email address,
%% and the form \ead[url] for the home page:
%% \title{Title\tnoteref{label1}}
%% \tnotetext[label1]{}
%% \author{Name\corref{cor1}\fnref{label2}}
%% \ead{email address}
%% \ead[url]{home page}
%% \fntext[label2]{}
%% \cortext[cor1]{}
%% \address{Address\fnref{label3}}
%% \fntext[label3]{}

\title{The Paris end of town? Urban typology through machine learning.} 




%% use optional labels to link authors explicitly to addresses:
\author[melb,monash]{Kerry A. Nice}
\corref{cor1}
\ead{kerry.nice@unimelb.edu.au}
\cortext[cor1]{Principal corresponding author}

\author[melb]{Jason Thompson} 
\author[melb]{Jasper Wijnands} 
\author[melb]{Gideon Aschwanden} 
\author[melb]{Claudia Pelizaro} 
\author[melb]{Mark Stevenson} 

\address[melb]{Transport, Health, and Urban Design Hub, Faculty of Architecture, Building, and Planning, University of Melbourne, Victoria 3010, Australia}
\address[monash]{School of Earth, Atmosphere and Environment, Monash University, Clayton, Australia}
\begin{abstract}





%\nomenclature{$T_{mrt}$}{mean radiant temperature (\SI{}{\degreeCelsius})}  
%\nomenclature{$UTCI$}{universal thermal climate index}
\end{abstract}

\begin{keyword}
machine learning, urban typology, CNTK

%% PACS codes here, in the form: \PACS code \sep code

%% MSC codes here, in the form: \MSC code \sep code
%% or \MSC[2008] code \sep code (2000 is the default)

\end{keyword}

\end{frontmatter}

%\begin{table*}[!t]   
%\begin{framed}
%\printnomenclature
%%\input{VTUF-3DDesign_Nomenclature}
%\end{framed}
%\end{table*}



%% \linenumbers

%% main text




\section{Introduction}\label{sec:introduction}



Introduction 
In 2013, 65\% of Australians lived in capital cities and it is estimated that this will increase to 72\% by 2053  (1). These projections are reflected in population growth estimates that will see the country’s population likely increase from 24 million today to over 33 million by 2050  (2). This growth is expected to mostly take place within the four largest capital cities (Sydney, Melbourne, Brisbane and Perth) with projections indicating that by 2053 both Sydney and Melbourne will have populations of almost 8 million  (2). As a consequence of this population growth, an array of transportation and population health challenges will emerge.

The importance of integrating transport plans with decisions regarding land-use is being recognised by governments (3). Land-use decisions significantly influence transport options and travel choice. Further, the high levels of low-density suburbanisation and high car dependency that characterise many Australian and New Zealand cities are remnants of a mid-twentieth century focus on home and car ownership (4). As a consequence, the status and practical utility of cycling or walking for daily travel requirements is limited (5). Low-density housing renders the cost of large-scale public transport prohibitive, producing a reliance on private vehicles and increasing exposure to risks associated with traffic speed, volume, emissions, and physical inactivity  (6, 7). Although some progress has been made in promoting active modes of transport the percentage of Australians walking or cycling to work remains low and primarily concentrated in low-speed, inner-city areas where distances are minimal. Public health policy-makers and urban/transport planners have an opportunity to change this situation by embracing strategies that proactively support active transport modes as facilitated by urban designs witnessed in other countries around the world.

For example, Copenhagen’s significant long-term investment in cycling infrastructure of over €30 million in cycling infrastructure for a city of approximately 550,000 residents has seen its rates of cycling mode share expand to 45\% of all trips without consequent increases in road trauma. Similarly, Amsterdam’s (and the Netherlands more generally) continued leadership in active transport promotion has demonstrated that cycling and walking can be re-claimed in areas and streets previously lost to car traffic, producing both health and economic benefit.

Substantial published evidence indicates that land-use affects transport modal choice and behaviour  (8-11). In addition, population health outcomes related to chronic conditions such as respiratory disease, cardiovascular disease, and Type 2 diabetes are strongly linked with more active transport modal choices  (12). Therefore, the creation of cities and precincts that facilitate greater uptake of safe, active transport could produce manifold environmental, productivity, and population health benefits. However, understanding the association between urban design features and health or environmental outcomes remains difficult, especially when underlying data, locations, methods, and demographics upon which statistical models are built vary considerably across studies. 
The confluence of recent advances in availability of geospatial information, computing power, and artificial intelligence offers new opportunities to understand how and where our cities differ and also, how they are alike. Departing from a traditional ‘top-down’ analysis of urban design features, this project analyses thousands of images of urban form from each Australian state and territory capital city, before making comparisons to a group of 20 other high profile cities from around the world including London, Paris, Copenhagen, Beijing, Los Angeles, and New York. We then map these international cities across each Australian example to produce within-city locations that demonstrate the closest connections between these areas and their international comparators. Finally, we suggest relationships between urban design and consequent transport, environmental, and population health outcomes these comparisons suggest may be generated by each location.

This work demonstrates an entirely new and highly efficient method for understanding the relationship between cities around the world, and the health, transport, and environmental consequences of their design. Perhaps most importantly, it also answers the age-old question, “Is there really a ‘Paris-end’ of your city?”.



\section*{References}\label{sec:ref}
%% If you have bibdatabase file and want bibtex to generate the
%% bibitems, please use
%%
  \bibliographystyle{elsarticle-harv} 
  \bibliography{library}

%% else use the following coding to input the bibitems directly in the
%% TeX file.

\begin{thebibliography}{00}

%% \bibitem[Author(year)]{label}
%% Text of bibliographic item

\bibitem[ ()]{}

\end{thebibliography}


%% The Appendices part is started with the command \appendix;
%% appendix sections are then done as normal sections
\appendix
\setcounter{table}{0}
\renewcommand{\thetable}{A\arabic{table}}

%\subsection{}                               %% Appendix A1, A2, etc.




\end{document}

\endinput
%%
%% End of file `elsarticle-template-harv.tex'.
